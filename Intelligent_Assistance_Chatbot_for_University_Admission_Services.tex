\documentclass[conference]{IEEEtran}
\IEEEoverridecommandlockouts
% The preceding line is only needed to identify funding in the first footnote. If that is unneeded, please comment it out.
\usepackage{cite}
\usepackage{url}
\usepackage{amsmath,amssymb,amsfonts}
\usepackage{algorithmic}
\usepackage{graphicx}
\usepackage{parskip}
\usepackage{textcomp}
\usepackage{xcolor}
\def\BibTeX{{\rm B\kern-.05em{\sc i\kern-.025em b}\kern-.08em
    T\kern-.1667em\lower.7ex\hbox{E}\kern-.125emX}}
    %\date{}  
    \setlength{\parindent}{1em} 
\begin{document}
\title{Intelligent Assistance Chatbot for University Admission Services\\
{\footnotesize \textsuperscript{}}
}

\author{\IEEEauthorblockN{Siska Restu Anggraeny Iskandar}\IEEEauthorblockA{\textit{Magister Informatika } \\
\textit{UIN Sunan Kalijaga}\\
Yogyakarta, Indonesia\\
18206050017@student.uin-suka.ac.id}

\and
\IEEEauthorblockN{Maria Ulfah Siregar}
\IEEEauthorblockA{\textit{Magister Informatika } \\
\textit{UIN Sunan Kalijaga}\\
Yogyakarta, Indonesia\\
maria.siregar@uin-suka.ac.id}

}
\maketitle
\begin{abstract}
Perubahan teknologi informasi menuntut seluruh sektor usaha termasuk sektor pendidikan untuk melakukan perubahan termasuk perubahan dalam sistem pelayanan. Dalam pelayanan perguruan tinggi termasuk pelayanan penerimaan mahasiswa perlu selalu memberikan pelayanan prima untuk menjamin kepuasan calon mahasiswa. Untuk memperoleh kepuasan calon mahasiswa selain dari kualitas pendidikan juga harus dibarengi dengan pemberian layanan konsultasi dan informasi kepada mereka. Paper ini mengusulkan pengembangan Chatbot, implementasi Chatbot dan optimalisasi Chatbot  yang diambil dari berbagai literatur penelitian terdahulu. Oleh karena itu, dengan menggunakan penelitian ini universitas dapat mengembangkan Chatbots cerdas lebih lanjut untuk membantu calon siswa menemukan informasi yang mereka butuhkan tanpa menunggu jawaban dari staf penerimaan.
\end{abstract}
\begin{IEEEkeywords}
Artificial Intelligence, Service, Chatbot, Sustainability
\end{IEEEkeywords}

\section{Pendahuluan}
Kemajuan terbaru dalam teknologi telah mendorong munculnya dan proliferasi ChatBot, yang merupakan program komputer cerdas yang dapat berinteraksi dengan user. Chatbots dapat memainkan peran sebagai penasihat virtual yang menggunakan konsep sistem pengenalan suara otomatis \emph{automatic speech recognition systems}, pembelajaran mesin, dan kecerdasan buatan (AI) \cite{b1}. Rahman, Al Mamun and Islam in \cite{b1} telah menyajikan pengenalan singkat tentang teknologi chatbots berbasis cloud bersama dengan pemrograman chatbots dan tantangan pemrograman di Era chatbot saat ini dan masa depan. 
Penggunaan chatbot kini ditandai sebagai salah satu bentuk pelayanan yang diberikan perusahaan dalam memaksimalkan good service kepala seluruh pengguna. Seperti yang dilansir pala laman info komputer tahun 2019 beberapa layanan Chatbots berbasis cloud di Indonesia seperti Bjtech sebagai vendor dari (BNI, Coca-Cola), Vutura Chatbot Indonesia (Telkom University, Telkom Indonesia), Inmotion, Botika, Bahasa.ai, Kata.ai dan banyak lainnya. Chatbots (yang juga disebut "asisten digital") dapat mengobrol dengan pengguna dalam berbagai cara termasuk entitas berbasis teks, antarmuka pengguna suara, dan entitas percakapan yang diwujudkan \cite{b2}.

Layanan chatbot kini ditandai sebagai salah satu solusi dalam mengoptimalkan layanan bagi pengguna jasa atau produk. Tidak terlepas pada sektor pendidikan seperti perguruan tinggi. Perguruan tinggi perlu selalu memberikan pelayanan prima untuk menjamin kepuasan calon mahasiswa. Pelayanan ini dapat berbentuk informasi yang dibutuhkan oleh calon mahasiswa terkait informasi pendaftaran, biaya kuliah ataupun bagi mahasiswa aktif yang membutuhkan pelayanan serta informasi yang berkaitan langsung dengan studinya. Kurang optimalnya layanan berbasis web yang diberikan oleh universitas terkait respon atas pertanyaan yang muncul atau informasi yang dibutuhkan oleh mahasiswa menandakan buruk atau kurang optimalnya layanan. \cite{b3}.

Penelitian yang mengkaji tentang kecerdasan buatan berbasis chatbot sebelumnya telah banyak dilakukan (misalnya, Universitas Elon menggunakan AI untuk membantu siswa melacak kursus yang diambil sebelumnya dan membantu mereka menerapkan informasi tersebut ke perencanaan kursus mereka \cite{b4}. Georgia State University menggunakan Pounce, chatbot yang dibuat oleh AdmitHub, mengurangi pencairan musim panas hingga lebih dari 20\%\ dengan menjangkau siswa melalui teks ketika mereka belum menyelesaikan tugas pada tanggal tertentu (Page dan Gehlbach, 2018) \cite{b5}. Fan, Fought, dan Gahn \cite{b6}, menemukan bahwa chatbot pop-up mampu memberikan layanan cepat kepada pengguna yang frustasi. Memasukkan chatbot pop-up di halaman beranda web akan meningkatkan penggunaan referensi obrolan, meningkatkan pengalaman dan kepuasan pengguna. Hsu dkk. \cite{b7} Allergy Bot menyediakan pilihan makan bagi pengguna tanpa pertanyaan informasi yang berlebihan. Ini juga menyediakan menu restoran yang tersedia, menyederhanakan tugas respon dan waktu untuk penanya.Horzyk, et al.,  \cite{b8}  penerapan mekanisme self adaptive dalam bentuk chatbot yang berinteraksi dengan pembeli online sangat meningkatkan pengalaman pelanggan dan belanja. Negi, et al., \cite{b9}, Chatbot mampu memberikan informasi tentang tarif yang berbeda untuk pemesanan mobil, melakukan pemesanan/reservasi, dan memodifikasi lokasi penjemputan pada pemesanan yang ada. Lasek and Jessa \cite{b10} menemukan hotel yang menggunakan chatbot mengalami pertumbuhan penjualan. Holotescu \cite{b11} menemukan menerapkan MOOCBuddy (sebuah chatbot) menyediakan pelajar online dengan interaksi mendongeng terkait dengan informasi online. Also, Calvert in \cite{b12} penyebaran chatbots untuk menanggapi penanya memiliki keunggulan berbeda dibandingkan manusia, termasuk yang berikut chatbot tidak lelah, chatbot tidak terganggu oleh pertanyaan konyol, chatbot memberikan konsistensi layanan, dan output chatbot tidak berkurang seiring waktu. 

Melihat banyaknya kegunaan yang dihasilkan oleh chat pada berbagai macam lini usaha, perguruan tinggi tampaknya perlu untuk menerapkan sistem informasi layanan berbasis chatbot. Penggunaan teknologi ini dalam perguruan tinggi akan memberikan fakultas dan staf kemampuan untuk lebih efektif dan efisien saat berkomunikasi dengan calon mahasiswa ataupun mahasiswa aktif. Untuk itu paper ini akan membahas terkait kecerdasan buatan berbasis chatbot untuk optimalisasi layanan di perguruan tinggi.

\section{Literature Review}
\subsection{Artificial Intelligence}
Menurut Cheston dan Shock \cite{b13}, “Antarmuka percakapan memungkinkan user berinteraksi dengan layanan yang seringkali kompleks melalui pesan atau sesuatu yang mereka lakukan setiap hari”. Ada batasan untuk menggunakan jenis teknologi ini karena interaksi hanya dapat dilakukan sedalam pengetahuan user. Misalnya, teknologi kecerdasan buatan hanya bisa menjawab pertanyaan langsung seperti, “Kapan pendaftaran dimulai? Berapa biaya pendaftaran?” kemudian artificial intelligence (AI) percakapan hanya dapat merespons dengan jawaban dasar, tanggal.

Tingkat kecerdasan buatan berikutnya menghubungkan aspek kontekstual percakapan dengan kemampuan untuk menafsirkan kebutuhan pengguna yang tidak disebutkan. Dengan mengintegrasikan perilaku user, kecepatan kurikulum, dan kemajuan, AI dapat mengintervensi (mengantisipasi kebutuhan intervensi) dan "mendorong" user ke tindakan terbaik berikutnya atau merujuk user langkah berikutny \cite{b14}.

Kecerdasan buatan dalam konteks penelitian ini adalah chatbot yang digunakan untuk membantu admisi perguruan tinggi dalam memaksimalkan layanan bagi mahasiswa dan calon mahasiswa. Tingkat AI ini akan membantu siswa mempertahankan kecepatan menuju kelulusan, menyelesaikan dokumen yang diperlukan, dan banyak lagi. Chatbot juga dapat bermanfaat bagi institusi dengan menyediakan data yang dapat digunakan untuk penjadwalan, analisis program, dan pengambilan keputusan lain untuk membuat institusi lebih sukses.

\subsection{Chatbot}
Menurut Shawar dan Atwell \cite{b15}, chatbots adalah aplikasi obrolan yang didukung oleh kecerdasan buatan yang fungsinya berkisar dari menjawab pertanyaan sederhana hingga mengambil bagian dalam percakapan yang kompleks 
Tergantung pada jenisnya, chatbots dapat berpartisipasi dalam percakapan berbasis suara dan teks; dalam penelitian ini, penulis fokus pada yang terakhir. Aplikasi chatting mampu memberikan respon yang berbeda terhadap permintaan atau pertanyaan dari pengguna yang berbeda.

Menurut Molnar, \&\ Szuts, \cite{b17}  chatbots adalah program komputer yang mampu melakukan percakapan serupa dengan orang-orang. Chatbots sering digunakan untuk mengotomatisasi atau mengoptimalkan proses bisnis. Jenis chatbot berkisar dari yang sederhana hingga yang kompleks, tujuannya adalah untuk mengeksploitasi spektrum kecerdasan buatan yang luas. Bot sederhana menangani pesan dan permintaan dasar dari pengguna. Saat berkomunikasi dengan pengguna, algoritma ini memberikan respons yang telah diprogram sebelumnya untuk input yang diberikan sebagai output. Misalnya untuk pertanyaan tentang suatu produk atau pertanyaan informasi lainnya. Dalam penelitian ini chatbot yang dimaksud adalah pada layanan admisi perguruan tinggi. 

\subsection{PEAS (Performance, Environment, Actuator \&\ Sensor)}
PEAS merupakan singkatan dari Performance, Environment, Aktuator dan Sensor yang merupakan bagian  penuh dari agent. Agen merupakan segala sesuatu yang mampu melihat, mengartikan dan mengetahui lingkungannya melalui alat sensor (Sensors) dan bertindak (Acting) melalui bantuan alat aktuator \cite{b18}. Lebih lanjut Russell mengklasifikasikan sifat agent kedalam tiga bagian yakni, 1) Rasional yakni agent dinilai bertindak paling benar, 2) Autonomi, yakni suatu agen dapat melakukan tindakan untuk memodifikasi persepsi masa depan hingga memperoleh informasi, 3) reaktif, yakni agen dapat menyimpulkan aspek lingkungan yang tersembunyi sebelum melakukan tindakan yang selektif.

\section{Metodologi Penelitian}
Penelitian ini menggunakan pendekatan studi pustaka, dengan mengambil sampel pada karya ilmiah yang sudah dipublikasi. Penulis akan menampilkan hasil temuan serta rekomendasi model dari chatbot yang telah diimplementasikan pada penelitian sebelumnya. Pengambilan sampel pada penelitian ini dengan cara memanfaatkan artikel, paper yang dipublikasi pada laman \emph{Google Scholar}, kemudian paper di cek kembali pada \emph{Scimago} untuk dilihat indeks prestasi jurnalnya. Setelah sampel terkumpul langkah selanjutnya peneliti melakukan riset pustaka untuk kemudian dianalisis sistem adoption chatbot di beberapa universitas yang sudah menerapkan untuk kemudian diambil rekomendasi dan kesimpulan.

\section{Pembahasan}
Penelitian yang mengulas optimalisasi sistem layanan menggunakan Chatbot di berbagai sektor usaha sebelum nya sudah banyak dilakukan, misalnya Fan, et al., \cite{b6}, mengevaluasi kinerja chatbot pop-up yang disisipkan pada halaman web situs web medis. Hsu et al. \cite{b7}, merancang chatbot yang menyediakan informasi alergi makanan untuk restoran. Horzyk, Magierski, \&\ Miklaszewski \cite{b8}  menghadirkan chatbot yang bertindak sebagai asisten toko yang berinteraksi dengan dan mengenali ciri-ciri kepribadian pelanggan berdasarkan preferensi pencarian dan belanja. Negi, Joshie, Chalamallay and Subramaniam  \cite{b9} membangun sistem berorientasi tugas (chatbot) yang memungkinkan percakapan manusia-mesin yang menanggapi permintaan pelanggan. Lasek \&\ Jessa \cite{b10} membandingkan kinerja chatbot pada situs web hotel/wisma yang berbeda. Holotescu \cite{b11} menguji peran chatbots dalam meningkatkan pengalaman belajar dalam kursus online terbuka besar-besaran (MOOCs). memeriksa berbagai penggunaan robot, termasuk chatbots, dalam melakukan tugas berulang seperti menanggapi pertanyaan pelanggan.
Penggunaan Chatbot di tingkat universitas sangat disarankan oleh beberapa penelitian sebelumnya (seperti, \emph{et al}\cite{b19}; Almahari, Bell dan Merhi \cite{b20};  Gbenga\cite{b20}; Chandra \&\ Suyanto \cite{b21}; Patel, Parikh, Patel dan Patel \cite{b23}; Le Hoang Shu \emph{et al}\cite{b24}; Hien, Cuong, Nam, Nhung \&\ Thang \cite{b25}). Chatbot mampu mengoptimalkan layanan dan membuat pengguna puas tanpa harus menunggu respon dari karyawan \emph{et al}\cite{b19} dan Chandra \&\ Suyanto\cite{b22}. Hasil penelitian lainnya menyebutkan bahwa sistem \emph{Artificial Intelligence} akan menghasilkan lebih banyak data untuk memberikan gambaran yang lebih jelas tentang proses belajar mengajar, yang memungkinkan rekomendasi informasi yang lebih akurat, dan penggunaan AI ini mampu meningkatkan kualitas layanan \cite{b23}.
Hefny, et al., \cite{b26}, menemukan bahwa chatbot memiliki banyak keunggulan, salah satunya bertujuan membantu para pelamar  untuk masuk universitas dan lebih efektif, disamping itu chatbot juga dapat menghemat waktu dan sumber daya mahasiswa dan staf penerimaan. Selain itu, terbukti bahwa membangun chatbot yang dimotivasi oleh pengguna memiliki efek positif pada kepuasan pengguna dan adopsi teknologi. Jenis chatbot yang dikembangkan oleh Hefny, et al., \cite{b26} ini adalah dengan menambahkan dukungan bilingualisme sehingga pelamar dapat mengajukan pertanyaan dalam bahasa Inggris atau Arab, dan chatbot akan merespons sesuai dengan bahasa input. Kemudian Isyarat sosial ditambahkan untuk memberikan interaksi seperti manusia dengan pelamar. Dari Gbenga et al., \cite{b21},  peneliti menemukan bahwa dengan mengadopsi artificial intelligence berbasis chatbot mampu  mengurangi beban kerja petugas penerimaan karena chatbot dapat merespons informasi dasar sehingga mengurangi jumlah panggilan dan surat yang harus ditanggapi. Mengadopsi solusi ini akan meningkatkan kualitas dan penyampaian layanan yang efisien dan secara real-time di sektor pendidikan.


\subsection{Penggunaan Artificial Intelligence Chatbot Untuk Perguruan Tinggi}
Secara keseluruhan, penggunaan chatbot saat ini telah banyak diadopsi oleh berbagai macam universitas di dunia seperti Jerman \cite{b26}, \cite{b19}, Polandia \cite{b21}, Pakistan \cite{b28}, Nigeria \cite{b21} dan di Georgia State University \cite{b29}, berdasarkan hasil review implementasi chatbot, secara keseluruhan aplikasi dibuat dengan ketentuan sebagai berikut:
\begin{itemize}
\item Ketika berada di luar kampus fitur dan fungsionalitas ini dapat dilakukan saat pelamar berada di rumah. Fitur ini juga berfokus dalam memberikan jawaban atas pertanyaan-pertanyaan yang paling sering diajukan selama proses penerimaan..
\item Fitur dirancang dengan ketentuan seorang pelamar dapat menanyakan tentang informasi berbagai jurusan dan fasilitas di universitas.
\item Pertanyaan yang diajukan oleh pelamar saat berada di kantor penerimaan seperti pertanyaan terkait dokumen yang dibutuhkan, biaya, beasiswa, dan bantuan keuangan.
\item Selain itu chatbot dibuat dengan kemampuan dapat melacak informasi pelamar seperti nama mereka, jenis sekolah menengah, jurusan yang dipilih, dan kebangsaan. Oleh karena itu, fitur ini dapat memberikan tanggapan dan personalisasi yang sesuai dengan kebutuhan pelamar alih-alih meminta pelamar mengunjungi situs web universitas untuk mencari jawaban. 
\item Pelamar dapat bertanya kepada chatbot tentang tes masuk yang harus diambil sebelum mendaftar di universitas.  
\item Chatbot memberikan contoh pertanyaan kepada pelamar serta tujuan dan biaya tes.
\item Chatbot dibuat mampu menjawab pertanyaan terkait fasilitas yang tersedia dan kegiatan ekstrakurikuler yang tersedia di situs web.
\item Chatbot tidak hanya dapat menjawab pertanyaan proses penerimaan, tetapi juga bertindak sebagai penasehat akademik untuk pelamar baru. Ini memberikan bantuan kepada calon mahasiswa baru dengan memberitahu mereka tentang berbagai fakultas dan jurusan yang ditawarkan di universitas, kursus yang diambil di setiap jurusan per semester, serta peluang dari masing-masing program studi yang diambil. 
\item Chatbot dikembangkan dengan tujuan untuk membantu calon mahasiswa terkait pertanyaan penerimaan di universitas secara tepat waktu, dapat diandalkan dan efisien sehingga, meningkatkan sistem yang ada.
Pelayanan dengan menggunakan chatbot direkomendasikan tidak hanya berhenti pada layananan diluar kampus, namun juga mampu memberikan informasi lengkap pada calon mahasiswa yang langsung mengunjungi universitas, dalam hal ini membuat chatbot dapat memperhatikan beberapa fitur yang harus ditambahkan kedalam aplikasi misalnya petunjuk jalan, dapat berbentuk maps atau peta dari masing-masing gedung atau fakultas yang terdapat di universitas.
\end{itemize}
Pada bagian implementasi, hal yang perlu diperhatikan bagi setiap pengembangan program atau aplikasi chatbot adalah kamus bahasa yang akan dijadikan sebagai salah satu rujukan untuk menafsirkan bahasa yang digunakan oleh user. Perguruan tinggi dapat membatasi atau menambahkan jenis bahasa yang dapat dideteksi oleh chatbot. 

\section{Kesimpulan}
Penelitian ini merekomendasikan perguruan tinggi untuk mengadopsi chatbot sebagai salah satu strategi optimalisasi layanan untuk calon mahasiswa baru maupun mahasiswa aktif. Chatbot ditandai sebagai salah satu kecerdasan buatan yang memberikan keuntungan bagi sektor pendidikan, hal ini dapat dilihat dari kegunaan yang membantu pegawai untuk mengurangi beban kerjanya. Chatbot dikembangkan dengan tujuan untuk membantu calon mahasiswa terkait pertanyaan-pertanyaan umum terkait penerimaan di universitas secara tepat waktu seperti alur pendaftaran, biaya, jadwal test, program beasiswa, informasi program studi dan bentuk fasilitas atau informasi lainnya. Bagi mahasiswa aktif chatbot dapat membantu informasi umum terkait persyaratan pembayaran, identitas dan status kemahasiswaan, persyaratan kelulusan dan bentuk informasi lainnya.

\section*{Author Contribution}
Siska Iskandar conceived the presented idea. Maria Ulfah Siregar supervised the project. Siska wrote the manuscript with support from Maria Ulfah Siregar. All authors discussed the results and contributed to the final manuscript.

\begin{thebibliography}{00}
\bibitem{b1} A. M. Rahman, A. Al Mamun, and A. Islam, “Programming challenges of chatbot: Current and future prospective,” 5th IEEE Reg. 10 Humanit. Technol. Conf. 2017, R10-HTC 2017, vol. 2018-Janua, no. 5, pp. 75–78, 2018, doi: \url{10.1109/R10-HTC.2017.8288910}.
\bibitem{b2}Harms, J. G., Kucherbaev, P., Bozzon, A., Houben, G. J, “Approaches for dialog management in conversational agents”. IEEE Internet Computing, 23(2), 13-22, 2018.
\bibitem{b3}Toniuc, D., Groza, A, “Climebot: An argumentative agent for climate change”. In 2017 13th IEEE International Conference on Intelligent Computer Communication and Processing (ICCP) (pp. 63-70). IEEE, 2017.
\bibitem{b4}Gardner, L, “How AI is infiltrating every corner of the campus”. The Chronicle of Higher Education, 8(04), 2018.
\bibitem{b5}Nurshatayeva, A., Page, L. C., White, C. C., Gehlbach, H. (2021). Are Artificially Intelligent Conversational Chatbots Uniformly Effective in Reducing Summer Melt? Evidence from a Randomized Controlled Trial. Research in Higher Education, 62(3), 392-402.
\bibitem{b6}Fan, S. C., Fought, R. L., Gahn, P. C, “Adding a Feature: Can a Pop-Up Chat Box Enhance Virtual Reference Services? Medical Reference Services Quarterly”,  36(3), 220- 228, 2017.
\bibitem{b7}P. Hsu, J. Zhao, K. Liao, T. Liu, and C. Wang, “AllergyBot: A Chatbot technology intervention for young adults with food allergies Dining out,” Conf. Hum. Factors Comput. Syst. - Proc., vol. Part F1276, pp. 74–79, 2017, doi: \url{10.1145/3027063.3049270}.
\bibitem{b8}A. Horzyk, S. Magierski, and G. Miklaszewski, “An Intelligent Internet Shop-Assistant Recognizing a Customer Personality for Improving Man-Machine Interactions,” Recent Adv. Intell. Inf. Syst., pp. 13–26, 2009.
\bibitem{b9}S. Negi, S. Joshi, A. Chalamallay, and L. V. Subramaniam, “Automatically extracting dialog models from conversation transcripts,” Proc. - IEEE Int. Conf. Data Mining, ICDM, pp. 890–895, 2009, doi: \url{10.1109/ICDM.2009.113}.
\bibitem{b10}S. Lasek, M., Jessa, “Chatbots for customer services on hotel websites,” Inf. Syst. Manag., vol. 2, no. 2, pp. 146–158, 2013.
\bibitem{b11}C. Holotescu, “MOOCBuddy: a chatbot for personalized learning with MOOCs,” Rochi – Int. Conf. Human-Computer Interact., vol. 8, pp. 91–94, 2016, [Online]. Available: \url{www.matrixrom.ro}.
\bibitem{b12}P. Calvert, “Robots, the Quiet Workers, Are You Ready to Take Over?,” Public Libr. Q., vol. 36, no. 2, pp. 167–172, 2017, doi: \url{10.1080/01616846.2017.1275787}.
\bibitem{b13}A. and S. Cheston, “The AI-first student experience,” 2017. \url{https://er.educause.edu/articles/2017/6/the-ai-first-student-experience}.
\bibitem{b14}M. Barret et al., “Using Artificial Intelligence to Enhance Educational Opportunities and Student Services in Higher Education.,” Inq. J. Virginia Community Coll., vol. 22, no. 1, p. 11, 2019.
\bibitem{b15}B. A. Shawar and E. Atwell, “Using dialogue corpora to train a chatbot Bayan Abu Shawar School of Computing,” in Corpus Linguistics 2003 conference, 2003, pp. 681–690, [Online]. Available: \url{https://docs.oracle.com/es/solutions/learn-about-designing-chatbot/plan-your-chatbot-design1.html}.
\bibitem{b16}D. Joyner, “Squeezing the limeade: Policies and workflows for scalable online degrees,” 2018, \url{doi: 10.1145/3231644.3231649}.
\bibitem{b17}G. Molnar and Z. Szuts, “The Role of Chatbots in Formal Education,” SISY 2018 - IEEE 16th Int. Symp. Intell. Syst. Informatics, Proc., pp. 197–201, 2018, {doi: 10.1109/SISY.2018.8524609}.
\bibitem{b18}S. J. R. and P. Norvig, “Artificial Intelligence A Modern Approach Second Edition,” Artif. Intell., pp. 183–227, 1996, [Online]. Available: \url{http://www.sciencedirect.com/science/article/pii/B9780121619640500091}.
\bibitem{b19}H. Agus Santoso et al., “Dinus Intelligent Assistance (DINA) Chatbot for University Admission Services,” Proc. - 2018 Int. Semin. Appl. Technol. Inf. Commun. Creat. Technol. Hum. Life, iSemantic 2018, pp. 417–423, 2018, \url{doi: 10.1109/ISEMANTIC.2018.8549797}.
\bibitem{b20}F. A. J. Almahri, D. Bell, and M. Merhi, “Understanding Student Acceptance and Use of Chatbots in the United Kingdom Universities: A Structural Equation Modelling Approach,” 2020 6th IEEE Int. Conf. Inf. Manag. ICIM 2020, pp. 284–288, 2020, doi:\url{10.1109/ICIM49319.2020.244712}.
\bibitem{b21}H. O. Gbenga, Oluwatobi and T. Okedigba, “An Improved Rapid Response Model for University Admission Enquiry System Using Chatbot,” Int. J. Comput., vol. 38, no. 1, pp. 123–131, 2020, [Online]. Available:\url{https://www.researchgate.net/publication/342248071_An_Improved_Rapid_Response_Model_for_University_Admission_Enquiry_System_Using_Chatbot}.
\bibitem{b22}Y. W. Chandra and S. Suyanto, “Indonesian chatbot of university admission using a question answering system based on sequence-to-sequence model,” in Procedia Computer Science, 2019, vol. 157, pp. 367–374, doi: \url{10.1016/j.procs.2019.08.179}.
\bibitem{b23}N. P. Patel, D. R. Parikh, D. A. Patel, and R. R. Patel, “AI and Web-Based Human-Like Interactive University Chatbot (UNIBOT),” Proc. 3rd Int. Conf. Electron. Commun. Aerosp. Technol. ICECA 2019, pp. 148–150, 2019, doi: \url{10.1109/ICECA.2019.8822176}.
\bibitem{b24}L. H. Su, T. Dang-Huy, T. Thi-Yen-Linh, N. Thi-Duyen-Ngoc, L. Bao-Tuyen, and N. Ha-Phuong-Truc, “Development of an AI Chatbot to Support Admissions and Career Guidance for Universities,” Int. J. Emerg. Multidiscip. Res., vol. 4, no. 2, pp. 11–17, 2020.
\bibitem{b25}H. T. Hien, P. N. Cuong, L. N. H. Nam, H. L. T. K. Nhung, and L. D. Thang, “Intelligent assistants in higher-education environments: The FIT-EBOt, a chatbot for administrative and learning support,” ACM Int. Conf. Proceeding Ser., pp. 69–76, 2018, doi: \url{10.1145/3287921.3287937}.
\bibitem{b26}W. El Hefny, Y. Mansy, M. Abdallah, and S. Abdennadher, “Jooka: A Bilingual Chatbot for University Admission,” Adv. Intell. Syst. Comput., vol. 1367 AISC, pp. 671–681, 2021, doi:\url{10.1007/978-3-030-72660-7_64}.
\bibitem{b27}A. Stachowicz-Stanusch and W. Amann, “Artificial Intelligence At Universities in Poland.,” Organ. Manag. Q., vol. 42, no. 2, pp. 63–82, 2018, [Online]. Available: \url{http://10.0.113.191/1899-6116.2018.42.6\%\0Ahttp://search.ebscohost.com/login.aspx?direct=true&AuthType=sso&db=bth&AN=133706868&lang=es&site=ehost-live}.
\bibitem{b28}Z. M. Khan, H. U. Rehman, M. Maqsood, and K. Mehmood, “Artificial Intelligence Based University Chatbot using Machine Learning,” Pakistan J. Eng. Technol., vol. 4, no. 2, pp. 108–112, 2021, [Online]. Available: \url{https://hpej.net/journals/pakjet/issue/view/78}.
\bibitem{b29}J. A. Boeding, “How Ai-Powered Chatbots Are Being Adopted and Used By Higher Education Institutions To Improve the Student Experience By Scaling Professional Staff,” University of Pennsylvania, 2020.
\end{thebibliography}

%\vspace{12pt}
%\color{red}
%IEEE conference templates contain guidance text for composing and formatting conference papers. Please ensure that all template text is removed from your conference paper prior to submission to the conference. Failure to remove the template text from your paper may result in your paper not being published.

\end{document}
